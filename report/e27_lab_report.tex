\documentclass[a4paper,10pt,twocolumn]{article}
\usepackage[latin1]{inputenc}
\usepackage[english]{babel}
\usepackage{amsmath}
\usepackage{amsfonts}
\usepackage{amssymb}
\usepackage{titling}
\usepackage{nomencl}
\usepackage{booktabs}
\usepackage{multicol}
\usepackage[font={footnotesize}]{caption}
\usepackage{graphicx}
\usepackage[style=ieee,backend=bibtex]{biblatex}
\usepackage{xspace}
\usepackage{fancyhdr}
\usepackage{varioref}
\usepackage{color}
\usepackage{siunitx}
\usepackage{multicol}
\usepackage[activate={true,nocompatibility},final,tracking=true,kerning=true,spacing=true,factor=1100,stretch=10,shrink=10]{microtype}
% activate={true,nocompatibility} - activate protrusion and expansion
% final - enable microtype; use "draft" to disable
% tracking=true, kerning=true, spacing=true - activate these techniques
% factor=1100 - add 10% to the protrusion amount (default is 1000)
% stretch=10, shrink=10 - reduce stretchability/shrinkability (default is 20/20)

% Reduce tracking around small caps to 40%
\SetTracking{encoding={*}, shape=sc}{40}

% Document info.
\author{Russell Maguire}
\title{Regression Assignment}
\date{\today}

% Path to images.
\graphicspath{{img/}}

% Setup nomenclature.
\makenomenclature
\setlength{\nomitemsep}{-0.1\parsep}
\renewcommand\nomgroup[1]{
    \ifthenelse{\equal{#1}{A}}{
        \item[\textbf{Acronyms}]}{
    \ifthenelse{\equal{#1}{B}}{
        \item[\textbf{Variables}]}{
    \ifthenelse{\equal{#1}{C}}{
        \item[\textbf{Constants}]}{
    \ifthenelse{\equal{#1}{D}}{
        \item[\textbf{Derived Constants}]}{
}}}}}

\newcommand{\nomunit}[2]{
    \renewcommand{\nomentryend}{
        \hspace*{\fill}\makebox[5em][r]{#1}\hspace{1em}\makebox[5em][l]{#2}
    }
}

% Setup SI units.

% Setup bibiliography.

% Header and footer.
\pagestyle{fancy}
\fancyhf{}
\lhead{\thetitle}
\rhead{\theauthor}
\cfoot{\thepage}
\renewcommand{\headrulewidth}{0pt}
\renewcommand{\footrulewidth}{0pt}

\begin{document}
    
% Title page.
\begin{titlepage}
    \centering
    \vspace*{\fill}
    \includegraphics[width=0.5\textwidth]{Durham}\\
    \vspace*{\fill}
    \LARGE\thetitle\\
    \large\theauthor\\
    \large L2 Electrical Engineering\\
    \large\thedate\\
    \vspace*{\fill}
\end{titlepage}

% Contents page.
\onecolumn
\pagenumbering{roman}
    
\tableofcontents
    
% Acronyms
\nomenclature[A0]{EMF}{Electromotive force.}
\nomenclature[A1]{RPM}{Revolutions per minute.}
\nomenclature[AA]{}{}

% Variables
\nomenclature[B0]{$E$}{Motor back EMF.
    \nomunit{}{\si{\volt}}}
\nomenclature[B1]{$I_a$}{Armature current.
    \nomunit{}{\si{\ampere}}}
\nomenclature[B2]{$R_{st}$}{Motor starter resistance.
    \nomunit{}{\si{\ohm}}}
\nomenclature[B3]{$N$}{Motor speed.
    \nomunit{}{RPM}}
\nomenclature[B4]{$P$}{Motor supply power.
    \nomunit{}{\si{\watt}}}
\nomenclature[B5]{$T$}{Motor torque.
    \nomunit{}{\si{\newton\metre}}}
\nomenclature[B6]{$\omega$}{Motor speed.
    \nomunit{}{\si{\radian\per\second}}}
\nomenclature[BB]{}{}

% Constants
\nomenclature[C0]{$B$}{Electric hoist load torque constant.
    \nomunit{0.247969}{\si{\newton\metre\per\radian\second}}}
\nomenclature[C0]{$L_{af}$}{Field-armature coupling inductance. 
    \nomunit{1.65}{\si{\henry}}}
%\nomenclature[C1]{$N_n$}{Nominal motor speed.
%    \nomunit{1220}{RPM}}
\nomenclature[C2]{$R_a$}{Armature resistance.
    \nomunit{1.6}{\si{\ohm}}}
\nomenclature[C3]{$R_f$}{Field resistance.
    \nomunit{240}{\si{\ohm}}}
\nomenclature[C4]{$T_0$}{Initial motor torque.
    \nomunit{45}{\si{\newton\metre}}}
\nomenclature[C5]{$V$}{Motor supply voltage.
    \nomunit{240}{\si{\volt}}}
\nomenclature[C6]{$V_f$}{Field supply voltage.
    \nomunit{240}{\si{\volt}}}
\nomenclature[C7]{$\omega_0$}{Initial motor speed.
    \nomunit{1}{\si{\radian\per\second}}}
\nomenclature[CC]{}{}

% Derived constants.
\nomenclature[D0]{$I_f$}{Field current.
    \nomunit{}{\si{\ampere}}}
\nomenclature[D1]{$K$}{Motor constant.
    \nomunit{}{\si{\newton\metre\per\ampere}}}
\nomenclature[DD]{}{}
   
\printnomenclature

% Main matter.
\twocolumn
\pagenumbering{arabic}

\section{Introduction}


\section{Background}

The most general form of DC motor is a separately excited DC motor, where the 
field and armature coils both have their own independent power supply.

Figure~\vref{fig:Circuit} contains the circuit diagram of a seperately excited 
DC motor, with a variable resistor for starting the motor without drawing too 
much current.

\begin{figure}[h]
    \centering
    \def\svgwidth{0.48\textwidth}
    \input{img/Circuit.pdf_tex}
    \caption{Circuit diagram of a seperately excited DC motor.}
    \label{fig:Circuit}
\end{figure}

\section{Methods}

\begin{table}
    \begin{tabular}{llrl}
        \toprule
        \textbf{Armature Components}
        & $V$      & 240 & \si{\volt} \\
        & $R_a$    & 1.6 & \si{\ohm} \\
        & $L_a$    & 12  & \si{\milli\henry} \\
        \midrule
        \textbf{Field Components}
        & $V_f$    & 240  & \si{\volt} \\
        & $R_f$    & 240  & \si{\ohm} \\
        & $L_f$    & 120  & \si{\henry} \\
        & $L_{af}$ & 1.65 & \si{\henry} \\
        \bottomrule
        
        
    \end{tabular}
    
\end{table}

\section{Results and Discussion}


\section{Conclusion}


% References.
\printbibliography

\clearpage

% Appendices.

\end{document}
